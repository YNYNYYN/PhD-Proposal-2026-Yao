This research develops an integrated digital twin system that combines structural health, microclimatic conditions and energy performance, providing comprehensive support for the operation and retrofit of historic buildings. The main contributions of the study can be summarised in the following three aspects:

\begin{enumerate}[noitemsep, topsep=2pt, leftmargin=*]
    \item \textbf{Development of a multidimensionally coupled real-time closed-loop regulation mechanism:} This study proposes a comprehensive digital twin framework for historic buildings that integrates structural health, microclimate and energy performance within a unified analytical and decision-making system. In doing so, it moves beyond existing research that predominantly focuses on single dimensions or isolated technical components.
    
    \item \textbf{Development of a quantitative evaluation extension tool for energy retrofitting of historic buildings:} Based on digital twin data and automated energy simulation, this research provides a robust and comprehensive quantitative assessment tool for the energy retrofitting of historic buildings, addressing the limitations of existing evaluation approaches in terms of comparability and dimensional completeness.
    
    \item \textbf{Establishment of a multi-source dataset integrating structure, microclimate and energy performance:} The multidimensional open-source dataset established in this study not only supports evidence-based decision-making within the present research, but also provides a valuable foundation for future studies, including algorithm development, model calibration and digital heritage conservation.
\end{enumerate}