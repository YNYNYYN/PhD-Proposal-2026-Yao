The preservation of historic buildings is vital to the continuity of cultural heritage \cite{ref1}, yet these buildings frequently encounter a range of challenges, including structural ageing, limited occupant comfort and high levels of energy use \cite{ref2, ref3}. The emergence of digital twin technology offers a means of addressing these issues \cite{ref4, ref5}. A digital twin is a continuously updated virtual counterpart of a physical asset \cite{ref6, ref7}, and its application to historic buildings enables real-time analysis, prediction and optimisation of structural conditions, indoor environments and energy use, thereby supporting their preservation and ongoing operation \cite{ref4, ref5, ref8, ref9}.

In terms of the development of virtual models for digital twins, existing research has investigated virtual modelling techniques as well as the processing of point cloud data \cite{ref10, ref11, ref12}. At the architectural level, relevant research primarily centres on data integration and the efficient computing of digital twin architectures. This includes achieving the interconnection of BIM and IoT via semantic knowledge graphs \cite{ref13, ref14}, and constructing lightweight model conversion processes suitable for web-based visualisation and simulation \cite{ref14, ref15, ref16}.

In the domain of structural health, relevant research is advancing from mere structural monitoring towards prediction and simulation verification. This encompasses the construction of preventive conservation systems \cite{ref4, ref5}, the utilisation of equivalent frames and finite element models for `physics-enhanced' analysis and calibration \cite{ref17, ref18, ref19}, collapse mechanism simulation based on ANSYS and Blender \cite{ref20}, the use of AI and photogrammetry for the identification of cracks and component degradation \cite{ref21, ref22, ref23}, and the detection of micro-cracks via laser scanning intensity data \cite{ref24}.

Regarding energy and environmental control, some studies compensate for the limitations of hardware monitoring by introducing virtual sensing and spatial prediction algorithms \cite{ref25, ref26}. Simultaneously, other research utilises advanced machine learning models to optimise energy use prediction \cite{ref27, ref28, ref29}, further integrating occupant localisation, behaviour recognition, and physiological indicators to realise human-centric intelligent environmental control \cite{ref30, ref31, ref32, ref33}.

Furthermore, energy retrofitting targeting historic buildings has gradually developed in recent years. Currently, several auxiliary tools have been developed, such as HiBERtool developed by Eurac Research in Europe, which established a knowledge base for retrofit solutions based on historic building characteristics \cite{ref34}. Additionally, some studies have utilised HBIM and Autodesk Insight to quantify energy-saving retrofits for historic buildings \cite{ref35}.

Although the studies above have advanced the field, existing digital twin research often remains confined to a single dimension or isolated technical component and lacks an integrated system that brings together structural health, microclimate and energy dimensions, a limitation that is particularly evident in the context of historic buildings. As a result, a significant gap remains in supporting long-term interventions for such buildings. In daily operation, there is no real-time control mechanism grounded in the coupled relationships between structure, microclimate and energy. In addition, when longer term energy retrofit interventions are considered, current tools are largely static databases focused on individual retrofit measures or discrete single case simulations, without the capacity to undertake quantitative evaluation of multiple scenarios automatically at the building scale.