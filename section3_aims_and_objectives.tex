To address the gaps in existing research, this study aims to develop a comprehensive digital twin system for historic buildings that enables real-time monitoring, intelligent regulation and predictive feedback across the structural, microclimate and energy dimensions. This overarching aim is operationalised through the following four specific objectives:

\begin{enumerate}[noitemsep, topsep=2pt, leftmargin=*]
    \item \textbf{Establishment of a multi-dimensional integrated Digital Twin platform:} To develop a digital twin architecture capable of integrating heterogeneous data related to structural health, environmental conditions, and energy systems, enabling real-time interaction between physical assets and their virtual counterparts.

    \item \textbf{Realisation of coupled intelligent regulation during operation:} To develop intelligent algorithms based on multidimensional data analysis to dynamically regulate HVAC and lighting systems, optimising energy use while ensuring structural safety and occupant comfort.

    \item \textbf{Development of a decision support tool for energy retrofitting:} To construct an extension tool that enables automated simulation of long-term energy retrofit scenarios, supporting multi-objective quantitative evaluation across economic cost, energy performance, and carbon emissions.

    \item \textbf{To establish a multi-source dataset:} To establish an open-source dataset incorporating information on structural health, occupant behaviour, microclimatic conditions, energy use, and retrofit-related parameters, covering real-time monitoring, predictive analysis, and simulation results. As the system is deployed in practice, newly generated data may further enrich and expand the dataset.
\end{enumerate}