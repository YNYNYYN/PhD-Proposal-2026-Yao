
Fig.~\ref{fig:framework} provides an integrated overview of the multi-domain digital twin system developed in this study, outlining the structural health, microclimate, and energy workflows, the open access dataset, and the case study verification pathway.

\begin{figure}[htbp]
    \centering

    \includegraphics[width=\textwidth]{1_methodology.png}
    \caption{The Framework of the Digital Twin System}
    \label{fig:framework}
\end{figure}

\subsection{Structural Health Monitoring and Analysis}
Historic buildings are subject to the combined effects of material ageing, environmental erosion, and service loads during long-term use \cite{ref41}, making structural safety a core risk factor. Moreover, traditional manual inspections struggle to identify the progressive degradation of structural performance and fail to provide deep insights into internal stress variations \cite{ref42}. Therefore, this section proposes constructing a structural health management system integrating real-time monitoring, early warning, and prediction, thereby achieving scientific, timely, and proactive protection of historic building structures.

\subsubsection{Real-time Structural Health Monitoring and Early Warning}
Settlement, inclination, and cracking are common structural issues in buildings \cite{ref43, ref44}. Therefore, this study intends to conduct real-time monitoring focusing on these three indicators. For settlement monitoring, hydrostatic levellers will be employed, with sensors deployed on key load-bearing masonry columns \cite{ref45}. Inclination monitoring will utilise tilt sensors, which should ideally be installed at the building's corners \cite{ref46}. For existing cracks, vibrating wire single-point displacement meters are proposed to monitor deep structural displacement, with sensors installed on both sides of the cracks or joints \cite{ref47}; for potential cracks, photogrammetric monitoring combined with AI algorithms will be used for identification and tracking \cite{ref22, ref23}.

When the monitoring system detects abnormal variations in parameters related to settlement, inclination, or cracks, an early warning mechanism will be automatically triggered to alert managers to conduct timely inspections or take intervention measures.

\subsubsection{Structural Stress Analysis}
Based on the Digital Twin BIM model, this study will discretise the building into a Finite Element Model (FEM) according to its structural characteristics, aiming to balance computational accuracy with analysis efficiency \cite{ref17, ref18, ref19}. It is planned to use MATLAB-based algorithms to process data from BIM to generate the FEM and employ algorithms for automatic correction based on actual conditions. This allows for the calculation of modal frequencies and mode shapes, serving as a basis for the automatic adjustment of the model using monitoring data \cite{ref18}. Modal analysis can calibrate material properties and model accuracy by comparing the model-predicted natural frequencies with on-site measurements; meanwhile, the predicted mode shapes (modal displacements) can provide a basis for the placement of monitoring equipment. Furthermore, stress analysis is used to evaluate structural safety and force distribution, further interpreting and predicting potential damage, as well as simulating the effects of intervention measures \cite{ref17}.

\subsection{Microclimate Monitoring and Optimisation}
Indoor air quality (IAQ) and a suitable hygrothermal environment are of great significance for historic buildings. On the one hand, reasonable temperature, humidity, and fresh air supply are fundamental conditions for guaranteeing occupant health and comfort \cite{ref48, ref49, ref50}. On the other hand, maintaining a stable and suitable hygrothermal environment can effectively inhibit mould growth \cite{ref5}, slow down material degradation, and preserve structural safety \cite{ref51, ref52}. To this end, this study proposes constructing a microclimate optimisation system encompassing real-time monitoring, occupancy activity identification, and predictive control, ensuring that the microclimate in all spaces of the historic building is continuously maintained within an appropriate range.

\subsubsection{Real-time Microclimate Monitoring}
This study intends to deploy sensors for temperature, relative humidity, CO$_2$ concentration, PM1.0 (particulate matter with a diameter less than or equal to 1 $\mu$m), light intensity \cite{ref5}, and wind speed \cite{ref15} in key spaces of the building to achieve real-time monitoring of the microclimate environment.

\subsubsection{Indoor Occupancy Activity and Pattern Identification} \label{sec:occupancy}
Occupancy activity significantly impacts the indoor microclimate \cite{ref53, ref54}, therefore, real-time awareness of occupant location is crucial. The system adopts an indoor positioning method based on Bluetooth Low Energy (BLE) beacons. By deploying a beacon network inside the building and utilising signals received by smart devices carried by occupants, continuous monitoring of occupancy status across different spaces is achieved. The system analyses the Received Signal Strength Indicator (RSSI) data and employs machine learning models to identify feature fingerprints of each area, thereby obtaining the real-time occupancy distribution within the building \cite{ref15, ref30}.

\subsubsection{Microclimate Prediction Models and Intelligent Control}
Due to the structural and heritage protection constraints of historic buildings, sensors often cannot be installed in positions that best reflect the true indoor environment. Furthermore, the distribution of indoor heat, moisture, and air pollutants exhibits significant non-uniformity, making it difficult for data obtained from real-time sensors to fully represent the overall microclimate condition. Additionally, traditional HVAC systems display obvious lag during the regulation process, making it difficult to meet the requirements of historic buildings for a stable thermal-humid environment.

To overcome these limitations, this study proposes using Gated Recurrent Unit (GRU) models to perform time-series forecasting of future indoor air quality and temperature, based on available real sensor data. Offline CFD simulations using Ansys Fluent are utilised to systematically analyse the indoor flow field, temperature and humidity distribution, and pollutant diffusion patterns, thereby identifying risk points in sensor monitoring blind spots. Ultimately, this allows for the calculation of the maximum risk value for the entire room using single-point future prediction data \cite{ref26}. A training set is constructed using real sensor data and multiple CFD simulations, and a fast prediction model is established using Radial Basis Functions (RBF) and neural networks. Upon inputting time-series data from sensors such as temperature, humidity, illuminance, and wind speed, this model can directly infer the future temperature and illuminance distribution of each space \cite{ref30}.

Based on the prediction results of the GRU and RBF models and the highest-risk areas identified by CFD simulation, assuming the building HVAC system possesses intelligent control capabilities, the VAV fresh air volume can be adjusted in advance according to predicted air quality trends \cite{ref55}; this ensures that IAQ remains consistently within the standard range. With regard to temperature and humidity regulation, cooling or dehumidification loads can be proactively adjusted based on algorithmically identified indoor worst-case locations, such as high humidity zones, in order to reduce the risk of local environmental exceedance. At the same time, the system should actively regulate lighting intensity so that it remains within an appropriate range in occupied areas \cite{ref30}. Through the aforementioned prediction-driven regulation strategies, the microclimate in all building spaces can be continuously maintained within an optimal range.

\subsection{Energy Performance Monitoring and Optimisation}
Energy use directly determines carbon emissions and has a significant impact on climate change \cite{ref36, ref37}. Reducing energy use and carbon emissions in the building sector is therefore critical to advancing sustainable development. Accordingly, this section proposes an integrated framework encompassing energy monitoring, intelligent optimisation, and prediction of post-retrofit energy performance, aimed at supporting energy management and decision making for historic buildings.

\subsubsection{Energy Use Monitoring}
Where conditions permit, smart meters will be deployed to obtain real-time electricity consumption data at the building level. For specific standalone equipment where smart meter installation is not feasible due to ageing wiring or spatial constraints, plug-in power meters will be used as an alternative to monitor the electricity consumption of major equipment and systems.

\subsubsection{Energy and Carbon Emission Prediction and Intelligent Optimisation}
Carbon emissions will be estimated using IPCC emission factors \cite{ref15}. Based on previously collected environmental monitoring data, EnergyPlus will be employed to conduct energy simulations in order to predict future building energy use. Using the real-time occupancy activity data obtained as described in Section \ref{sec:occupancy}, control strategies will be implemented for HVAC and lighting systems. The system will automatically reduce or shut down equipment loads in unoccupied areas according to the spatial and temporal distribution of occupants.

\subsubsection{Energy Retrofit Performance Prediction Tool}
This research also proposes the development of an extended tool for evaluating energy retrofitting strategies for historic buildings, with the objective of quantifying energy use, carbon emissions, and economic costs under different retrofit scenarios. The retrofit measures considered include the replacement of windows with improved airtightness using double or triple glazing, the addition of organic, petrochemical, or mineral insulation layers to roofs, walls, and floors \cite{ref38}, the installation of photovoltaic panels, and the replacement of existing systems with high efficiency heat pumps and advanced HVAC systems.

The tool is developed in Python and implements a fully automated workflow designed to enable efficient building energy simulation using the EnergyPlus engine. It integrates previously collected real-time data such as occupancy activity and HVAC system information together with geometric model data. By invoking Python libraries such as Eppy, the tool enables execution from parametric modification of IDF model files to local simulation runs \cite{ref39, ref40}. At the operational level, users only need to select the retrofit option and input the local electricity tariff via a terminal interface. The system then automatically executes the simulation and saves the annual energy use prediction results to a specified directory. Built in algorithms subsequently calculate annual carbon emissions and economic cost savings based on IPCC emission factors.

Due to technical limitations, as well as the complexity and parameter sensitivity inherent in building energy simulation, this tool is primarily intended for users with professional expertise in EnergyPlus, in order to ensure the accuracy and reliability of the simulation results.

\subsection{Multidimensional Datasets}
Data from real-time monitoring, prediction, and simulation data from the structural, microclimatic, and energy dimensions will be unified into datasets and made openly available to support decision making and future research.

The structural health dataset includes time series data from structural monitoring such as settlement, inclination, and crack displacement, crack imagery, and finite element analysis results together with corresponding mechanical parameters. Beyond current safety assessment, this dataset will support future research on automated identification of structural deterioration patterns, seismic response analysis of historic buildings, and the simulation and validation of structural strengthening schemes.

The occupancy activity and microclimate dataset integrates multi-source information, including high resolution occupancy behaviour data such as Bluetooth fingerprinting and spatiotemporal distributions, indoor and outdoor environmental monitoring data covering temperature, humidity, IAQ, illuminance, and wind speed, as well as flow fields and pollutant dispersion patterns derived from CFD simulations. Occupancy behaviour is highly stochastic and constitutes a decisive factor in energy use variation and indoor environmental fluctuations. Its inherent unpredictability has long constrained the accuracy of building environmental simulations \cite{ref56, ref57}. The occupancy activity dataset constructed in this study provides essential baseline data for addressing the long-standing challenge of modelling user behaviour and also supports the optimisation of indoor microclimate prediction models.

The energy dataset integrates real-time energy use monitoring data with future energy simulation outputs. These data provide a quantitative basis for establishing long-term energy performance benchmarks, managing carbon emissions, and optimising energy use in historic buildings.

Through cross dimensional data integration and open access, this database provides robust long-term support for analysing complex internal coupling mechanisms, optimising machine learning model training, implementing predictive maintenance, and ensuring the continuous evolution of the digital twin system.

\subsection{Overall System Architecture}
To provide a clear overview of the proposed system, Fig.~\ref{fig:tech_implementation} illustrates the overall architecture of the digital twin framework. The framework is designed to address the challenges of high computational complexity and stringent real time requirements in historic building monitoring. It comprises HBIM Foundation Layer, Physical Sensing and Actuation Layer, Edge Computing Layer, Cloud Data and Service Layer, and Communication Network Layer.

\begin{figure}[htbp]
    \centering

    \includegraphics[width=\textwidth]{2_architecture.png}
    \caption{Technical Implementation of the Digital Twin System}
    \label{fig:tech_implementation}
\end{figure}

Within the HBIM Foundation Layer, laser scanning, photogrammetry, and parametric modelling are employed to construct a semantic building information model compliant with IFC and HBIM standards, providing a unified geometric and semantic foundation for the system \cite{ref10, ref11, ref12, ref15}.

At the Physical Sensing and Actuation Layer, distributed nodes for structural health monitoring, microclimatic environmental monitoring, and energy use monitoring are developed based on STM32. These nodes are equipped with multiple interfaces to support RS485, UART, I$^2$C, analogue signals, and PWM control, enabling multimodal data acquisition and execution control across the structural, microclimate, and energy domains \cite{ref58, ref59, ref60, ref61, ref62, ref63, ref64, ref65, ref66, ref67, ref68, ref69, ref70, ref71, ref72, ref73, ref74}.

The Edge Computing Layer adopts Raspberry Pi as an intermediate tier, integrating models trained on high performance workstations \cite{ref18, ref26}. This layer is responsible for data parsing, protocol conversion, local AI inference, real-time control, and edge data buffering, thereby maintaining low latency regulation even under conditions of cloud unavailability or network instability \cite{ref5, ref68, ref75, ref76, ref77, ref78, ref79, ref80, ref81, ref82}.

Within the Cloud Data and Service Layer, the system employs a hybrid architecture combining an MQTT Broker, InfluxDB, and MySQL to enable unified management, association, and retrieval of time series monitoring data and static semantic information \cite{ref15, ref83, ref84}. Finally, a digital twin visualisation interface based on WebGL2 and BIMServer dynamically couples IFC models with cleaned sensor data, enabling cross device access to three-dimensional real-time visualisation and interactive functions \cite{ref15, ref30, ref85}. The Communication Network Layer supports multipath and lightweight data transmission through UART, RS485, Bluetooth, Wi-Fi, and MQTT, ensuring stable uploading of high frequency sensor data and real-time interaction among distributed devices \cite{ref30, ref79, ref80, ref86, ref87, ref88}.

\subsection{Case Study}
Finally, a real building will be selected to validate the feasibility of the proposed framework under real world conditions. This validation will examine the stability of sensor data, the accuracy of structural and environmental prediction models, and the effectiveness of the coupled regulation mechanisms. In addition, by applying the retrofit extension tool to the selected building, simulation results from different retrofit scenarios can be compared to assess the applicability of the tool in practical building contexts. The overall validation process will ensure that the proposed framework demonstrates both scientific rigour and practical operability.