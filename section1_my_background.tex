My academic background and research interests lie in the fields of building environmental performance analysis, architectural digital technologies, and historic building conservation.
I have received formal training in architecture and the sustainable built environment.
I have developed substantial research experience in architectural digital technologies, building energy simulation, and data-driven analytical methods.
Against this background, I am applying to the PhD programme at the University of Sydney with the aim of developing a comprehensive digital twin framework specifically designed for historic buildings, integrating structure, microclimate, and energy analysis.

I am currently pursuing a Master's degree in Architecture at Central South University, China, with a primary research focus on historic buildings.
Throughout my studies, I have consistently achieved strong academic results and have been awarded several academic scholarships.
I completed an MSc in Sustainable Building Studies at the University of Sheffield and hold a Bachelor's degree in Engineering.
This interdisciplinary educational background spanning architecture, engineering, and sustainable building enables me to engage simultaneously with architectural space and historical context, building physics and environmental performance, as well as digital systems and data structures.
It provides a well-aligned intellectual and methodological foundation for doctoral research centred on digital twins that integrate microclimate and energy considerations.

My research experience can be broadly categorised into two main areas: architectural digital technologies and sustainable building research.

\vspace{1em}
\noindent \textbf{Architectural Digital Technologies}

\begin{enumerate}[noitemsep, topsep=2pt, leftmargin=*]

    \item Digital Restoration and Display of Confucius Temples in Hunan Province during the Qing Dynasty (MEng Thesis).

    \begin{itemize}[noitemsep, topsep=1pt, leftmargin=0.6em]
        \item[] Objective: To digitise and systematically present seventeen existing Qing-Dynasty Confucius temples in Hunan Province.
    \end{itemize}

    \begin{itemize}[noitemsep, topsep=0pt, leftmargin=1.5em]
        \item Employing HBIM technology, 3D scanning, high-definition imaging, and Revit software to reconstruct accurate digital models.
        \item Recording and analysing current structural conditions and architectural details for heritage preservation and research purposes.
        \item Restoring missing architectural elements and historical states across different periods using available textual and visual materials.
    \end{itemize}

    \item Digital Twin for Sizheng Society Historic Site (Changsha, China).

    \begin{itemize}[noitemsep, topsep=1pt, leftmargin=0.6em]
        \item[] Objective: To achieve the preventive protection of a brick and timber structure historic building by conducting real-time monitoring of its structural condition.
    \end{itemize}

    \begin{itemize}[noitemsep, topsep=0pt, leftmargin=1.5em]
        \item Integrating multi-sensor data into a digital twin platform for visualisation, anomaly detection, and deterioration analysis.
        \item Monitoring structural behaviour including inclination, settlement, and crack development.
    \end{itemize}

\end{enumerate}

\vspace{1em}
\noindent \textbf{Building Environmental Performance}

\begin{enumerate}[noitemsep, topsep=2pt, leftmargin=*]

    \item LCA and LCC Analysis for Retrofitting of Hongda Dwelling.

    \begin{itemize}[noitemsep, topsep=1pt, leftmargin=0.6em]
        \item[] Objective: To develop a dynamic LCA--LCC framework for evaluating energy retrofit strategies for vernacular buildings, using a rammed earth dwelling in Liuyang, China as a case study.
    \end{itemize}

    \begin{itemize}[noitemsep, topsep=0pt, leftmargin=1.5em]
        \item Utilised EnergyPlus with Python (eppy) for automated batch simulation execution, and employed Python scripts for data processing and comparative analysis.
        \item Integrated dynamic factors including future climate scenarios, grid decarbonisation, electricity price variations, and inflation.
        \item Assessed envelope insulation materials and low-carbon technologies, analysing their individual and synergistic effects.
        \item Applied Pareto optimisation and Marginal Abatement Effectiveness (MAE) analysis to identify cost-effective retrofit solutions.
    \end{itemize}

    \item Translating the Design Principles of Yinzi Dwellings in China Into Contemporary Energy-Efficient Housing (MSc Thesis).

    \begin{itemize}[noitemsep, topsep=1pt, leftmargin=0.6em]
        \item[] Objective: To evaluate the passive design principles of Yinzi dwellings and adapt their effective vernacular features to contemporary housing to improve energy efficiency.
    \end{itemize}

    \begin{itemize}[noitemsep, topsep=0pt, leftmargin=1.5em]
        \item Conducted building performance simulations using DesignBuilder and EnergyPlus.
        \item Developed advanced courtyard modelling approaches in DesignBuilder for complex historic Chinese buildings.
        \item Employed Python scripts for data extraction, processing, and comparative analysis.
    \end{itemize}
\end{enumerate}

These research experiences provide a strong foundation for my proposed PhD project.
In addition, my undergraduate training in engineering, with a specialisation in mechanisation and automation, equips me with the technical competence required for hardware architecture design, sensor system integration, and data acquisition and processing workflows.
This background is particularly relevant to the development of a comprehensive digital twin framework tailored to historic buildings.

Further details of my academic background and research experience are provided in the accompanying CV.